\documentclass[12pt]{article}

\usepackage{times}
\usepackage[hidelinks]{hyperref}   \usepackage{setspace}

\begin{document}

\onehalfspacing

\title{The Importance of Digitalization}
\author{Rami Nakkar}
\date{\today}
\maketitle

\begin{abstract}
asdfasdfasdfasdf
\end{abstract}

\newpage

\section{Introduction}
Digitalization has become a ubiquitous term in today's world, referring to the use of digital technologies to transform business models, create new revenue streams, and improve efficiency (Marr, 2018). The impact of digitization is far-reaching, affecting various aspects of our lives, including the economy, society, and individuals. From the rise of e-commerce to the proliferation of social media platforms, digitalization has changed the way we interact with each other and conduct business (Heng, 2019). In this context, it is important to understand the implications of digitalization and how it is shaping our future. This requires a nuanced understanding of its benefits and challenges, as well as its potential impact on different stakeholders (World Economic Forum, 2016). In this essay, we will explore the impact of digitalization on the economy, society, and the individual.

\section{The Concept and Process of Digitalization}
Defining digitalization is a complex task due to its broad and multidimensional nature \cite{JasonBloomberg.2018}. Marcus Heidlund and Leif Sundberg, for instance, emphasize the public sector perspective in their definition, describing digitalization as 'the intensified use of digital technologies in the public sector' \cite{Heidlund.2023}. This perspective, while valid, only scratches the surface of the transformative scope and depth that digitalization brings across various sectors and areas of life. \\
While defining digitalization can be seen as a theoretical exercise, it is in the process of digitalization that we witness the tangible manifestations of these definitions. Understanding digitalization as  the networking of all areas of the economy and society as proposed by Daniel R. A. Schallmo \cite{Schallmo.}, necessitates examining how this intensification unfolds and affects various aspects of our lives. This leads us to delve into the process of digitalization, which encompasses the phases and shifts that accompany the adoption and integration of digital technologies. \\
The process of digitalization encompasses several critical steps, including the adoption and integration of digital technologies \cite{GablerWirtschaftslexikon.5312023}, fostering digital literacy \cite{vanAckeren.2019}, effecting cultural change \cite{Nevmatulina.2022}, and managing the resultant data effectively \cite{DORDUNCU.2021}. This transformation permeates various sectors and influences our economy, society, and individual lives \cite{HenningHummert.2018}, making digitalization a comprehensive and ongoing process.

\section{The Impact of Digitalization on the Economy}
asdfasdfasdfasdf

\section{The Impact of Digitalization on Society}
asdfasdfasdfasdf

\section{The Impact of Digitalization on the Individual}
asdfasdfasdfsadf

\section{Conclusion}
asdfasdfasdfasdf

\newpage

\bibliographystyle{plain} % We choose the "plain" reference style
\bibliography{refs} % Entries are in the refs.bib file

\end{document}
